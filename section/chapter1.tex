\chapter{Pangkat \textit{HIMATIF} \textit{(Himpunan Mahasiswa Teknik Informatika)}}
\par
\paragraph{}Pangkat \textit{HIMATIF} merupakan kedudukan yang menunjukkan tingkatan seorang anggota himpunan dalam susunan organisasi \textit{HIMATIF (Himpunan Mahasiwa Teknik Informatika)}. Setiap anggota himpunan memiliki pangkat yang bertujuan untuk mendeskripsikan tingkatan kemampuan yang dimiliki dan pengabdian yang diberikan. 

\section{Deskripsi Pangkat}
Pangkat \textit{HIMATIF} terdiri dari Pangkat 1 yang disebut sebagai Junior, Pangkat 2 yang disebut sebagai Senior, dan Pangkat 3 yang disebut sebagai Instruktur.

\section{Tujuan}
Berikut Tujuan diadakannya Pangkat \textit{HIMATIF}:
\begin{enumerate}
 \item Sebagai tolok ukur atas kemampuan seorang anggota himpunan dalam Bidang yang ditekuni (Teknik Informatika).
 \item Mengetahui seberapa besar pengabdian seorang anggota himpunan kepada organisasi \textit{HIMATIF}.
 \item Sebagai motivator dalam meningkatkan pengetahuan dan kreativitas di bidang keilmuan (Teknik Informatika).
 \item Meningkatkan kerja sama antar sesama anggota himpunan. \textit{HIMATIF}.
\end{enumerate}
