\chapter{Kenaikan Pangkat}
\paragraph{}Kenaikan pangkat merupakan suatu penghargaan yang diberikan atas kemampuan dan pengabdian anggota himpunan terhadap organisasi \textit{HIMATIF}. Kenaikan pangkat dimaksudkan agar anggota himpunan \textit{HIMATIF} mampu meningkatkan kemampuan dan produktivitasnya, memiliki motivasi yang lebih untuk menciptakan suatu inovasi.

\section{Syarat dan Ketentuan Naik Pangkat}
Syarat dan ketentuan kenaikan pangkat telah dimusyawarahkan dan disetujui oleh beberapa orang dosen yang terlibat dalam organisasi \textit{HIMATIF}. Berikut syarat dan ketentuan yang harus dilaksanakan oleh anggota himpunan \textit{HIMATIF}.
Syarat dan Ketentuan untuk Mendapatkan Pangkat 1 (Junior):
\begin{enumerate}
 \item Mengikuti Morris dan mengikuti pelatihan sebanyak 1 kali dalam bidang TI.
 \item Membuat draft jurnal nasional.
 \item Membuat buku Ber-ISBN berhalaman 80 - 100 lembar.
 \item Penilaian Pembina.
 \item Aktif sebagai kepengurusan Himpunan.
\end{enumerate}
Syarat dan Ketentuan untuk Mendapatkan Pangkat 2 (Senior):
\begin{enumerate}
 \item Menjadi instruktur pelatihan sebanyak 2 kali dalam bidang TI.
 \item Mengikuti PKM (Pengabdian Kepada Masyarakat) sebanyak 1 kali.
 \item Mengikuti pelatihan bidang TI sebanyak 3 kali.
 \item Penilaian pembina.
 \item Aktif sebagai kepengurusan himpunan.
\end{enumerate}
Syarat dan Ketentuan untuk Memperoleh Pangkat 3 (Instruktur):
\begin{enumerate}
 \item Menjadi instruktur pelatihan sebanyak 4 kali.
 \item Mengikuti PKM (Program Kreativitas Mahasiswa) sebanyak 1 kali.
 \item Mengikuti perlombaan IT , terdapat bukti sertifikat mengikuti lomba.
 \item Membuat product inovasi.
 \item Penelitian kaloborasi dosen sebanyak 1 kali.
 \item Penilaian pembina.
 \item Aktif sebagai kepengurusan himpunan.
\end{enumerate}

\subsection{Morris}
\par
Morris merupakan kegitan masa orientasi yang dilakukan oleh jurusan teknik informatika. Tujuan dari Morris adalah untuk membangun karakter mahasiswa teknik informatika yang bermutu. Anggota himpunan harus mengikuti kegiatan Morris.

\subsection{Mengikuti Pelatihan}
\par
Mengikuti pelatihan merupakan bukti dari anggota yang telah melaksanakan dengan dibuktikannya sertifikat pelatihan yang telah dikikutinya.

\subsection{Menerbitkan Buku Ber-ISBN}
\par
Seorang anggota himpunan akan diakui kemampuannya apabila telah menerbitkan buku ber-isbn. Penerbitan ini bermaksud agar anggota himpunan dapat berbagai ilmu yang dimiliki kepada orang lain. Ilmu tidak akan ada gunanya apabila digunakan untuk diri sendiri. Ilmu akan bermanfaat apabila seseorang dapat membagikan ilmunya tersebut kepada orang lain. Inilah tujuan dari seorang anggota himpunan wajib menerbitkan buku ber-isbn.

\subsection{Draf Jurnal Nasional}
\par
Jurnal merupakan tulisan khusus yang memuat artikel suatu bidang ilmu tertentu. jurnal dibuat oleh seseorang yang berkompeten di bidangnya dan diterbitkan oleh suatu instansi.\\
\par
Tujuan anggota himpunan dapat membuat sebuah jurnal yaitu agar seorang anggota himpunan dapat menunjukkan kemampuannya kepada orang lain. Apabila seorang anggota himpunan menciptakan suatu produk dan ingin agar produknya tersebut diketahui oleh dunia, maka melalui jurnal anggota himpunan dapat memperkenalkan kemampuan dan produk-produk ciptaannya.

\subsection{Pembina Himpunan}
\par
Pembina Himpunan merupakan penilaian mingguan yang akan dilaksanakan oleh seluruh anggota himpunan HIMATIF. Pembina Himpunan bertujuan agar dapat mengevaluasi kinerja para anggota himpunan setiap minggunya dan memusyawarahkan kegiatan-kegiatan yang akan dilaksanakan untuk meningkatkan kinerja para anggota himpunan HIMATIF. Pembina Himpunan sangat berpengaruh terhadap kenaikan pangkat, baik pangkat 1, pangkat 2, maupun pangkat 3. Dengan adanya Pembina Himpunan diharapkan para anggota himpunan HIMATIF dapat menilai kemampuan masing-masing dan termotivasi untuk lebih giat dalam meningkatkan kemampuan.

\subsection{Aktif Sebagai Kepengurusan Himpunan}
\par
Anggota himpunan harus aktif dalam organisasi, penilaian tersebut akan dinilai oleh Ketua Himpunan (KAHIM).

\subsection{Instruktur Pelatihan}
\par
Instruktur pelatihan adalah seseorang yang mengadakan pelatihan yang berkaitan dengan bidang keilmuannya dengan tujuan untuk membagikan ilmu yang diperoleh kepada orang lain.\\
\par 
Seorang anggota himpunan dituntut untuk bisa menjadi instruktur pelatihan, dengan tujuan agar anggota himpunan dapat membantu orang lain dalam belajar dan memahami sesuatu hingga orang yang mengikuti pelatihan dapat memahami segala sesuatu yang diajarkan oleh anggota himpunan dengan baik.

\subsection{PKM (Program Kreativitas Mahasiswa)}
\par
PKM (Program Kreativitas Mahasiswa) merupakan kegiatan yang dibentuk oleh Direktorat Jendral Pembelajaran dan Kemahasiswaan Kementrian Riset sebagai suatu wadah untuk menampung kreativitas dan inovasi para mahasiswa berdasarkan Ilmu Sains dan Teknologi.
Jenis PKM (Program Kreativitas Mahasiswa) yang dapat diikuti:
\begin{enumerate}
 \item PKM-Pengabdian Kepada Masyarakat (PKM-M)
 \item PKM-Penerapan Teknologi (PKM-T)
 \item PKM-Karsa Cipta (PKM-KC)
 \item PKM-Artikel Ilmiah (PKM-AI)
 \item PKM-Gagasan Tertulis (PKM-GT)
\end{enumerate}
\par
PKM akan memberikan dampak yang sangat baik bagi seorang anggota himpunan. Ketika melakukan kegiatan PKM-M, anggota himpunan dituntut untuk bisa membagikan ilmunya kepada orang lain. Selain itu, kemampuan anggota himpunan dalam berbicara di depan umum akan lebih terasah, sehingga tidak ada lagi anggota himpunan yang tidak bisa menjadi seorang \textit{public speaker}. Begitupun dengan PKM lainnya, PKM-T yang bertujuan untuk membuat anggota himpunan lebih mahir di bidang teknologi. PKM-KC yang bertujuan agar seorang anggota himpunan dapat menciptakan suatu produk dari masalah yang ada dan produk yang memiliki daya jual tinggi. PKM-AI yang bertujuan sama halnya seperti penerbitan jurnal yaitu agar seorang anggota himpunan dapat menulis sebuah karya dari produk-produk yang diciptakan dan memperlihatkan hasil karyanya kepada dunia.

\subsection{Product Inovasi}
\par
Produk inovasi merupakan sebuah proses yang memberikasn solusi dengan permasalahan yang ada. Sehingga anggota himpunan harus mencipatan sebuah produk inovasi yang mahal dan berkualitas maupun murah dan berkualitas. 

\subsection{Mengikuti Perlombaan}
\par
Anggota himpunan harus mengikuti perlombaan dalam bidang IT dan bersertifikat, agar dapat menunjang pengetahuan anggota himpunan dalam bidang IT.

\subsection{Penelitaian Kolaborasi Dosen}
\par
Penelitian Kolaborasi dapat dilakukan dengan Dosen, agar anggota himpunan memiliki bekal untuk mendalami project.