\chapter{Form Penilaian Kenaikan Pangkat HIMATIF}
\par
Form penilaian kenaikan pangkat HIMATIF bertujuan untuk mengevaluasi kegiatan anggota HIMATIF setiap minggunya, sehingga setiap kegiatan yang dilakukan anggota HIMATIF akan dinilai dan dijadikan pertimbangan untuk kenaikan pangkat. penilaian ini akan didiskusikan oleh para anggota himpunan setiap minggunya.\\
Berikut gambaran Form Penilaian Kenaikan Pangkat HIMATIF:
Form Penilaian Kenaikan Pangkat 1 (Junior):\\
Form penilaian pangkat 1 (Junior) berisi syarat dan ketentuan yang telah ditetapkan untuk kenaikan pangkat 1.\\
\begin{enumerate}
 \item Mengikuti Morris .
 \item Mengikuti pelatihan sebanyak 1 kali dalam bidang TI.
 \item Membuat draft jurnal nasional.
 \item Membuat buku Ber-ISBN berhalaman 80 - 100 lembar (bagian isi). Anggota himpunan akan dinilai berdasarkan buku yang telah diterbitkan dan dinilai berdasarkan kinerja anggota himpunan dalam mengerjakan buku yang akan diterbitkan.
 \item Penilaian pembina merupakan penilaian mingguan yang diberikan oleh Pembina kepada anggota himpunan setiap minggunya.
 \item Aktif sebagai kepengurusan Himpunan.
 
\end{enumerate}
Form Penilaian Kenaikan Pangkat 2 (Senior):\\
Form penilaian pangkat 2 (Senior) berisi syarat dan ketentuan yang telah ditetapkan untuk kenaikan pangkat 2.\\
\begin{enumerate}
 \item Menjadi instruktur pelatihan sebanyak 2 kali. Apabila anggota himpunan mengadakan pelatihan dan menjadi instruktur pelatihan sebanyak 2 kali, maka anggota himpunan akan dievaluasi berdasarkan kinerja anggota himpunan saat menjadi instruktur. Penilaian anggota himpunan akan dinilai dan dirata-ratakan sehingga memperoleh nilai (0-100) dan indeks nilai (A-E).
 \item Mengikuti PKM (Pengabdian Kepada Masyarakat) atau pelatihan. Apabila seorang anggota himpunan telah melakukan PKM atau pelatihan sebanyak 1 kali, maka anggota himpunan akan dinilai sesuai dengan kinerja yang telah ia lakukan pada saat PKM atau pelatihan berlangsung. Penilaian berupa angka 0-100 yang nantinya akan di rata-ratakan, hingga memperoleh indeks mutu (A-E) sesuai dengan nilai
 \item Mengikuti Pelatihan Bidang IT.
 \item Penilaian pembina. Penilaian pembina merupakan penilaian mingguan yang diberikan oleh Pembina kepada anggota himpunan setiap minggunya.
 \item Aktif sebagai kepengurusan Himpunan.
 
\end{enumerate}
Form Penilaian Kenaikan Pangkat 3 (Instruktur):\\
Form penilaian pangkat 3 (Instruktur) berisi syarat dan ketentuan yang telah ditetapkan untuk kenaikan pangkat 3.\\
\begin{enumerate}
 \item Menjadi instruktur pelatihan sebanyak 2 kali. Apabila anggota himpunan mengadakan pelatihan dan menjadi instruktur pelatihan sebanyak 4 kali, maka anggota himpunan akan dievaluasi berdasarkan kinerja anggota himpunan saat menjadi instruktur. Penilaian anggota himpunan akan dinilai dan dirata-ratakan sehingga memperoleh nilai (0-100) dan indeks nilai (A-E).
 \item Mengikuti PKM (Program Kegiatan Mahasiswa). Apabila seorang anggota himpunan telah melakukan PKM atau pelatihan sebanyak 1 kali, maka anggota himpunan akan dinilai sesuai dengan kinerja yang telah ia lakukan pada saat PKM berlangsung. Penilaian berupa angka 0-100 yang nantinya akan di rata-ratakan, hingga memperoleh indeks mutu (A-E) sesuai dengan nilai.
 \item Membuat Produk Inovasi.
 \item Mengikuti perlombaan IT , terdapat bukti sertifikat mengikuti lomba.
 \item Penelitian Kolaborasi Dosen.
 \item Penilaian pembina. Penilaian pembina merupakan penilaian mingguan yang diberikan oleh Pembina kepada anggota himpunan setiap minggunya.
 \item Aktif sebagai kepengurusan Himpunan.
 
 
\end{enumerate}



